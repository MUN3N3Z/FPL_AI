Research at the intersection of artificial intelligence and fantasy sports has grown substantially over the past decade, with approaches ranging from statistical modeling to sophisticated reinforcement learning techniques.

Central to my methodology is the Bayesian modeling of player abilities. Dixon and Coles (1997) pioneered this field with their Bayesian model for football match outcomes, accounting for attacking and defensive strengths of teams while incorporating home advantage \cite{dixon1997}. Their model has become a foundation for football prediction systems and informed my fixture simulation component.

Foundational work on FPL team formation was done in the seminal paper by Matthews, Ramchurn, and Chalkiadakis (2012) establishing the first comprehensive AI framework for fantasy football management \cite{matthews2012}. Their work "Competing with Humans at Fantasy Football: Team Formation in Large Partially-Observable Domains" laid the groundwork for modeling the FPL team selection problem as a Belief-State Markov Decision Process. The authors demonstrated that an AI agent could perform at the top 0.2\% of approximately 2.5 million human competitors, despite functioning under uncertainty about player performances. Their approach utilized Bayesian Q-learning with Value of Perfect Information (VPI) exploration to balance immediate rewards with long-term planning. While groundbreaking, Matthews et al. relied heavily on the use of wildcard chips at predetermined gameweeks - 8 and 24 - to optimize team value and selection. This paper deliberately constraints the agent to operate without special chips, thereby testing its ability to navigate the FPL season through standard weekly transfers only. Furthermore, my model extends their approach by incorporating team-specific formations and training on a significantly larger dataset, spanning seven seasons (2016/17 - 2022/23) rather than just a single season.

Beyond reinforcement learning, numerous researchers have explored purely statistical approaches to fantasy sports optimization. Bonomo, Durán, and Marenco (2014) applied mathematical programming techniques to fantasy team selection, using integer linear programming to identify optimal squads within budget constraints \cite{bonomo2014}. Their model, while effective for single-gameweek optimization, lacked the capacity for sequential planning that reinforcement learning provides.