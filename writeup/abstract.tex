\thispagestyle{plain}
\begin{center}
    \Large
    \textbf{A CONSTRAINED REINFORCEMENT LEARNING AGENT FOR FANTASY FOOTBALL}
    
    \vspace{0.4cm}
    \large
    % Thesis Subtitle
    
    \vspace{0.4cm}
    \textbf{Tony Munene Kinyua}
    
    \vspace{0.9cm}
    \textbf{Abstract}
\end{center}

This research explores the viability of artificial intelligence in managing Fantasy Premier League (FPL) teams for individuals constrained by time differences, busy schedules, and the demands of active team management. Building upon Matthews, Ramchurn, and Chalkiadakis'(2013) groundbreaking work in AI-driven team formation, this study examines whether a reinforcement learning agent can compete with or outperform the average human FPL manager without utilizing special chips. The agent employs a Bayesian approach to model players' abilities amidst performance uncertainties, with priors informed by historical data spanning seven seasons (2016/17-2022/23). Formulating team selection as a Markov Decision Process solved through Bayesian Q-learning, this work addresses the social dimension of FPL competition while offering a solution for enthusiasts unable to commit to the rigorous demands of traditional team management. The research provides insights into automated decision-making in partially-observable, complex domains while preserving the competitive essence that makes fantasy sports engaging. [RESULTS]