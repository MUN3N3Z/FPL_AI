I set out to explore whether a reinforcement learning agent could effectively manage a Fantasy Premier League Team while operating under constraints that reflect the challenges faced by casual managers, particularly those in different time zones from the U.K. Specifically, I wanted to find out how such an agent would perform compared to the average FPL manager and whether it could achieve competitive results without replying on special chips.

Contrary to initial expectations, my constrained Bayesian Q-learning agent performed below the baseline average human managers across the season. The deliberate limitation of using only one free transfer per gameweek, while reflecting realistic constraints for casual managers, significantly hampered the agent's ability to adapt to the dynamic EPL environment. Therefore, the FPL agent struggled to recover from sub-optimal decisions made early in the season. The agent's performance particularly suffered during the irregular schedule periods, including Blank Gameweeks and Double Gameweeks. These periods require specialized strategies that proved difficult for our reinforcement learning approach to discover autonomously.
